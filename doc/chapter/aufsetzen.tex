\chapter{Aufsetzen der Dockerumgebung}
Für das Aufsetzen der Dockerumgebung wurde der Hochleistungsrechner der DHBW Bad Mergentheim von Professor Doktor Carsten Müller genutzt. Für das Aufsetzen wurde dazu mit den vorgegebenen Passwörtern eingeloggt. 
\section{Aufsetzen der Installationsumgebungen}
Wichtig ist es zu schauen, dabei ob der Rechner zur Kommunikation mit dem Internet ein Proxy benötigt. Wenn dies der Fall ist folgende Befehle eingeben:

\begin{verbatim}
    sudo nano /etc/apt/apt.conf
\end{verbatim}
Darin müssen dann folgende Zeilen hinzugefügt werden:
\begin{verbatim}
    Acquire::http::Proxy "http://<adress>:<port>";
    Acquire::https::Proxy "https://<adress>:<port>";
\end{verbatim}
Die zweite Zeile muss nur hinzugefügt werden, wenn die Verbindung über eine verschlüsselte Verbindung geschehen sollen. Für den Fall, dass kein Proxy genutzt werden sollte, allerdings einer genutzt wird, müssen statt der obigen Zeilen in \textit{/etc/apt/apt.conf} folgende Zeilen einsetzen:
\begin{verbatim}
    Acquire::http::Proxy "false";
    Acquire::https::Proxy "false";
\end{verbatim}

\section{Aufsetzen Docker}
Für den Docker Container aufsetzen kann im generellen der Generellen Dockerdokumentation gefolgt zu werden zu finden \textit{\href{https://docs.docker.com/desktop/install/ubuntu/}{hier}}. Es kann durchaus sein, dass nach dem Folgen der Schritte die Fehlermeldung kommt, dass das Repository nicht gefunden werden konnte. Wenn dies der Fall ist, die nötigen SChritte rückwerts gehen und die verschiedenen Dateien wieder löschen und dem \textit{\href{https://www.digitalocean.com/community/tutorials/how-to-install-and-use-docker-on-ubuntu-20-04}{zweiten Tutorial}} folgen. Dies hatte in dieser Instanz geholfen das Docker Repository hinzuzufügen.

